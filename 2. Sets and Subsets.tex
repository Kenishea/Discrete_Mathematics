\documentclass{article}
\usepackage{graphicx} % Required for inserting images

\title{Set Operations}
\author{Kenishea Antoniette Shettlewood}
\date{August 2023}

\begin{document}
\maketitle

To prove whether \(A\) is a subset of \(B\), you can show that every element of \(A\) is also in \(B\).



For every \(a \in A\), we can see that \(a\) is not necessarily even, which means \(a \notin B\).

Therefore, \(A\) is not a subset of \(B\).

\(\textbf{B} = \{2, 4, 6, 8\}\),
\(\textbf{C} = \{1, 3, 5, 7, 9\}\), and
\(\textbf{D} = \{3, 4, 5\}\).

We need to find a set \(\textbf{X}\) that satisfies the following conditions:

1. \(\textbf{X}\) and \(\textbf{B}\) are disjoint.
2. \(\textbf{X} \subseteq \textbf{D}\) but \(\textbf{X} \not\subseteq \textbf{B}\).
3. \(\textbf{X} \subseteq \textbf{A}\) but \(\textbf{X} \not\subseteq \textbf{C}\).

Let's construct \(\textbf{X}\) step by step:

- To satisfy condition 1, \(\textbf{X}\) must have no elements in common with \(\textbf{B}\), so \(\textbf{X} \cap \textbf{B} = \emptyset\).

- To satisfy condition 2, we can choose elements from \(\textbf{D}\) that are not in \(\textbf{B}\). For example, \(\textbf{X} = \{3, 5\}\) satisfies this condition because \(\textbf{X} \subseteq \textbf{D}\) (since it contains only elements from \(\textbf{D}\)) but \(\textbf{X} \not\subseteq \textbf{B}\) (since it contains elements that are not in \(\textbf{B}\)).

- To satisfy condition 3, we need to choose elements from \(\textbf{A}\) that are not in \(\textbf{C}\). For example, \(\textbf{X} = \{0, 2, 4, 6, 8, 9\}\) satisfies this condition because \(\textbf{X} \subseteq \textbf{A}\) (since it contains only elements from \(\textbf{A}\)) but \(\textbf{X} \not\subseteq \textbf{C}\) (since it contains elements that are not in \(\textbf{C}\)).

So, \(\textbf{X} = \{0, 2, 4, 6, 8, 9\}\) satisfies all three conditions.



\end{document}