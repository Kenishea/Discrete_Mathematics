\documentclass{article}
\usepackage{graphicx} % Required for inserting images

\title{Relations}
\author{Kenishea Antoniette Shettlewood}
\date{August 2023}

\begin{document}
\maketitle

1. **Find all \(a R b\)**:
   
   To find all \((a, b)\) in \(R\) (\(aRb\)), we simply list the pairs in the relation \(R\):

   \[
   R = \{(1, y), (1, z), (3, y), (4, x), (4, z)\}
   \]

2. **Find \(R^{-1}\) (the inverse of \(R\))**:
   
   To find the inverse of \(R\), we swap the elements of each pair:

   \[
   R^{-1} = \{(y, 1), (z, 1), (y, 3), (x, 4), (z, 4)\}
   \]

3. **Determine the domain and range of \(R\)**:
   
   - Domain (\(Dom(R)\)): The set of all first elements in the ordered pairs in \(R\):

   \[
   Dom(R) = \{1, 3, 4\}
   \]

   - Range (\(Ran(R)\)): The set of all second elements in the ordered pairs in \(R\):

   \[
   Ran(R) = \{x, y, z\}
   \]

4. **Check if \(R\) is a reflexive relation**:
   
   A relation \(R\) on a set \(A\) is reflexive if, for every element \(a\) in \(A\), \((a, a)\) is in \(R\).

   In this case, \(R\) is not reflexive because it does not contain pairs of the form \((a, a)\) for every element \(a\) in \(A\). For example, there is no \((2, 2)\) in \(R\).

5. **Check if \(R\) is a symmetric relation**:
   
   A relation \(R\) on a set \(A\) is symmetric if, for every \((a, b)\) in \(R\), \((b, a)\) is also in \(R\).

   In this case, \(R\) is not symmetric because, for example, it contains \((1, y)\) but does not contain \((y, 1)\), and it contains \((4, x)\) but does not contain \((x, 4)\).

\end{document}